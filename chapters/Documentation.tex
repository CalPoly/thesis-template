\chapter{Documentation}
\section{Scope}
In order for SINATRA to be used by other users and developer the code must be well documented. It is important to have systems of documentation in place for all of the different levels of instructions, guidelines, comments, and information. Those can be broken into two sections; for the user and for the developer. These systems must be simple, reliable, clear, and resilient. The goal is for the code to be easy to distribute, simple to learn as much as needed about how it works for that specific new purpose, and for the changes, bugs and suggestions to be centralized.
\section{Developers}
The system chosen for the developers is GitHub. It is a online file storage, syncing, and collaboration work space for developing code bases. It is easy to use, popular and therefore support for using it is strong, and it is very powerful. SINATRA is housed on GitHub by the author and is shared with Dr. Greig and Dr. Marshall. On account of the code's license the repository is private and developers can get access by contacting the Aerospace Department at Cal Poly, SLO. GitHub has three major features used in SINATRA; the commits system, the branches system, and the ReadME files. The commits system is a method of allowing multiple developers to work on the same code base without breaking the other developers builds. Each developer works on the code on their local machines. Once they have a stable addition to SINATRA, they commit it to GitHub. Whenever other users are working, they can pull those changes on to their local machine and developing moves along smoothly. It does require communication between the developers to ensure they aren't working on the same lines of code and creating different outcomes. 